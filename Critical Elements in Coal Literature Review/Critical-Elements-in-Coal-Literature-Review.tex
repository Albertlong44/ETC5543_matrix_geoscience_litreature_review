\documentclass[preprint, 3p,
authoryear]{elsarticle} %review=doublespace preprint=single 5p=2 column
%%% Begin My package additions %%%%%%%%%%%%%%%%%%%

\usepackage[hyphens]{url}

  \journal{International Journal of Coal Geology} % Sets Journal name

\usepackage{graphicx}
%%%%%%%%%%%%%%%% end my additions to header

\usepackage[T1]{fontenc}
\usepackage{lmodern}
\usepackage{amssymb,amsmath}
% TODO: Currently lineno needs to be loaded after amsmath because of conflict
% https://github.com/latex-lineno/lineno/issues/5
\usepackage{lineno} % add
\usepackage{ifxetex,ifluatex}
\usepackage{fixltx2e} % provides \textsubscript
% use upquote if available, for straight quotes in verbatim environments
\IfFileExists{upquote.sty}{\usepackage{upquote}}{}
\ifnum 0\ifxetex 1\fi\ifluatex 1\fi=0 % if pdftex
  \usepackage[utf8]{inputenc}
\else % if luatex or xelatex
  \usepackage{fontspec}
  \ifxetex
    \usepackage{xltxtra,xunicode}
  \fi
  \defaultfontfeatures{Mapping=tex-text,Scale=MatchLowercase}
  \newcommand{\euro}{€}
\fi
% use microtype if available
\IfFileExists{microtype.sty}{\usepackage{microtype}}{}
\usepackage[]{natbib}
\bibliographystyle{elsarticle-harv}

\ifxetex
  \usepackage[setpagesize=false, % page size defined by xetex
              unicode=false, % unicode breaks when used with xetex
              xetex]{hyperref}
\else
  \usepackage[unicode=true]{hyperref}
\fi
\hypersetup{breaklinks=true,
            bookmarks=true,
            pdfauthor={},
            pdftitle={Literature Review on Critical Elements in Coal},
            colorlinks=false,
            urlcolor=blue,
            linkcolor=magenta,
            pdfborder={0 0 0}}

\setcounter{secnumdepth}{5}
% Pandoc toggle for numbering sections (defaults to be off)


% tightlist command for lists without linebreak
\providecommand{\tightlist}{%
  \setlength{\itemsep}{0pt}\setlength{\parskip}{0pt}}







\begin{document}


\begin{frontmatter}

  \title{Literature Review on Critical Elements in Coal}
    \author[Matrix Geoscience]{Kane Maxwell%
  \corref{cor1}%
  }
   \ead{kane.maxwell@matrixgeoscience.com} 
    \author[Monash University]{Yuhao Long%
  %
  }
   \ead{ylon0012@student.monash.edu} 
    \author[Monash University]{Evan Ginting%
  %
  }
   \ead{egin0003@student.monash.edu} 
      \affiliation[Matrix Geoscience]{
    organization={Matrix Geoscience},}
    \affiliation[Monash University]{
    organization={Monash University Masters of Buisness Analytics},}
    \cortext[cor1]{Corresponding author}
  
  \begin{abstract}
  This paper summarises the literature on critical elements contained in
  coal. Specifically the paper adresses the following questions:

  \begin{enumerate}
  \def\labelenumi{\arabic{enumi})}
  \item
    What are critical elements, what are they used for and what is their
    projected demand?
  \item
    What types of critical elements are found in coal, and in what
    concentrations?
  \item
    What concentrations of critical elements would be considered
    economic to extract?
  \item
    Are there any existing coal mines, or coal basins that are
    extracting critical elements from coal, or coal tailings?
  \item
    What analytical and processing methods can be used to extract
    critical elements from coal?
  \end{enumerate}
  \end{abstract}
    \begin{keyword}
    Coal \sep Critical Elements \sep 
    Rare Earth Elements
  \end{keyword}
  
 \end{frontmatter}

\textbf{Instructional}

\emph{Delete this section after you have fully prepared this document}

In your answers to the questions in this literature review please use
high quality references (e.g peer reviewed Journal). References can be
added to the ``mybibfile.bib'' within the literature review folder. For
example see below a paragraph citing a reference.

``The Fort Cooper Coal Measures and associated stone bands (e.g.,
tuffaceous clays) have previously been identified as having potential to
contain elevated concentrations of rare earth elements (REE) and other
`strategic' elements of economic importance
\citep{Hodgkinson2020, Hodgkinson2021}.''

Please note that I have added a considerable number of potentially
relevant references to the .bib file; so check the bib file before
adding a new reference.

\hypertarget{critical-elements-overview}{%
\section{Critical elements overview}\label{critical-elements-overview}}

Answer the following question:

\begin{enumerate}
\def\labelenumi{\arabic{enumi}.}
\tightlist
\item
  What are critical elements (also referred to as critical minerals
  including rare earth elements)? What are they used for and what is
  their projected demand?
\end{enumerate}

In your answer, in addition to global references and demand, also try to
focus on their importance within Australia (there has been a lot of
press and recently announced funding for exploration).

You can use
\href{https://www.ga.gov.au/scientific-topics/minerals/critical-minerals}{Geoscience
Australia} to get a broad overview, but I expect that you will provide
`proper' peer reviewed Journal articles in your answer.

Consider using a table to summarize the results; for example listing the
Element, its uses and demand, and source of citation in a table format.

\hypertarget{critical-elements-in-coal}{%
\section{Critical elements in coal}\label{critical-elements-in-coal}}

Answer the following question:

\begin{enumerate}
\def\labelenumi{\arabic{enumi}.}
\setcounter{enumi}{1}
\tightlist
\item
  What types of critical elements are found in coal, and in what
  concentrations?
\end{enumerate}

For good background info relevant to Australia please start with the
ACARP report (C29030) by \citet{Hodgkinson2021}. Also, in our data
analysis we will be focusing on determining PAAS values - make sure you
know what this means and how to calculate this - it is fully explained
in \citet{Hodgkinson2021}.

\hypertarget{economic-concentrations}{%
\section{Economic concentrations}\label{economic-concentrations}}

Answer the following question:

\begin{enumerate}
\def\labelenumi{\arabic{enumi}.}
\setcounter{enumi}{2}
\tightlist
\item
  What concentrations of critical elements (including rare earth
  elements) would be considered economic to extract from coal?
\end{enumerate}

You might try the Report on Rare Earth Elements from Coal and Coal
Byproducts (\citet{usde2017}) to start with.

Also try \citet{Seredin2012}; \citet{Dai2016}; \citet{Qin2015b};
\citet{SUN2014}.

\citet{Eterigho2021}

\hypertarget{existing-economic-deposits}{%
\section{Existing economic deposits}\label{existing-economic-deposits}}

Answer the following question:

\begin{enumerate}
\def\labelenumi{\arabic{enumi}.}
\setcounter{enumi}{3}
\tightlist
\item
  Are there any existing coal mines, or coal basins that are extracting
  critical elements from coal, or coal tailings?
\end{enumerate}

Try \citet{Qin2015b} to start with. Also try \citet{Sun2010} and
\citet{Qin2015}

\hypertarget{extraction-of-critical-elements}{%
\section{Extraction of critical
elements}\label{extraction-of-critical-elements}}

Answer the following question:

\begin{enumerate}
\def\labelenumi{\arabic{enumi}.}
\setcounter{enumi}{4}
\tightlist
\item
  What analytical and processing methods can be used to extract critical
  elements from coal?
\end{enumerate}

Try \citet{Eterigho2021} for REE extraction example. Also look at
\citet{Qin2015} and Qin2015b

\renewcommand\refname{References}
\bibliography{mybibfile.bib}


\end{document}
