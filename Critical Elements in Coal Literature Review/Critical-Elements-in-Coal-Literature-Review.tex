\documentclass[preprint, 3p,
authoryear]{elsarticle} %review=doublespace preprint=single 5p=2 column
%%% Begin My package additions %%%%%%%%%%%%%%%%%%%

\usepackage[hyphens]{url}

  \journal{International Journal of Coal Geology} % Sets Journal name

\usepackage{lineno} % add

\usepackage{graphicx}
%%%%%%%%%%%%%%%% end my additions to header

\usepackage[T1]{fontenc}
\usepackage{lmodern}
\usepackage{amssymb,amsmath}
\usepackage{ifxetex,ifluatex}
\usepackage{fixltx2e} % provides \textsubscript
% use upquote if available, for straight quotes in verbatim environments
\IfFileExists{upquote.sty}{\usepackage{upquote}}{}
\ifnum 0\ifxetex 1\fi\ifluatex 1\fi=0 % if pdftex
  \usepackage[utf8]{inputenc}
\else % if luatex or xelatex
  \usepackage{fontspec}
  \ifxetex
    \usepackage{xltxtra,xunicode}
  \fi
  \defaultfontfeatures{Mapping=tex-text,Scale=MatchLowercase}
  \newcommand{\euro}{€}
\fi
% use microtype if available
\IfFileExists{microtype.sty}{\usepackage{microtype}}{}
\usepackage[]{natbib}
\bibliographystyle{elsarticle-harv}

\ifxetex
  \usepackage[setpagesize=false, % page size defined by xetex
              unicode=false, % unicode breaks when used with xetex
              xetex]{hyperref}
\else
  \usepackage[unicode=true]{hyperref}
\fi
\hypersetup{breaklinks=true,
            bookmarks=true,
            pdfauthor={},
            pdftitle={Literature Review on Critical Elements in Coal},
            colorlinks=false,
            urlcolor=blue,
            linkcolor=magenta,
            pdfborder={0 0 0}}

\setcounter{secnumdepth}{5}
% Pandoc toggle for numbering sections (defaults to be off)


% tightlist command for lists without linebreak
\providecommand{\tightlist}{%
  \setlength{\itemsep}{0pt}\setlength{\parskip}{0pt}}



\usepackage{booktabs}
\usepackage{caption}
\usepackage{longtable}
\usepackage{colortbl}
\usepackage{array}
\usepackage{anyfontsize}
\usepackage{multirow}



\begin{document}


\begin{frontmatter}

  \title{Literature Review on Critical Elements in Coal}
    \author[Matrix Geoscience]{Kane Maxwell%
  \corref{cor1}%
  }
   \ead{kane.maxwell@matrixgeoscience.com} 
    \author[Monash University]{Yuhao Long%
  %
  }
   \ead{ylon0012@student.monash.edu} 
    \author[Monash University]{Evan Ginting%
  %
  }
   \ead{egin0003@student.monash.edu} 
      \affiliation[Matrix Geoscience]{}
    \affiliation[Monash University]{}
    \cortext[cor1]{Corresponding author}
  
  \begin{abstract}
  This paper summarises the literature on critical elements contained in
  coal. Specifically the paper adresses the following questions:

  \begin{enumerate}
  \def\labelenumi{\arabic{enumi})}
  \item
    What are critical elements, what are they used for and what is their
    projected demand?
  \item
    What types of critical elements are found in coal, and in what
    concentrations?
  \item
    What concentrations of critical elements would be considered
    economic to extract?
  \item
    Are there any existing coal mines, or coal basins that are
    extracting critical elements from coal, or coal tailings?
  \item
    What analytical and processing methods can be used to extract
    critical elements from coal?
  \end{enumerate}
  \end{abstract}
    \begin{keyword}
    Coal \sep Critical Elements \sep 
    Rare Earth Elements
  \end{keyword}
  
 \end{frontmatter}

\hypertarget{critical-elements-overview}{%
\section{Critical elements overview}\label{critical-elements-overview}}

Answer the following question:

\begin{enumerate}
\def\labelenumi{\arabic{enumi}.}
\tightlist
\item
  What are critical elements (also referred to as critical minerals
  including rare earth elements)? What are they used for and what is
  their projected demand?
\end{enumerate}

As the burgeoning ideology of environmental sustainability and
geopolitical tension, The concept of critical elements received
paramount attentions in the contemporary global landscape. Critical
elements, or also referred as critical minerals, is vital metallic or
non-metallic element that underpin the functionality and advancement of
core technologies, economic frameworks, and national security but may be
susceptible for exogenous risk in supply chain.

According to
\href{https://www.industry.gov.au/publications/critical-minerals-strategy-2023-2030}{Critical
Minerals Strategy 2023--2030} issued by \citet{geoscience2023},There has
total 15 elements are listed as top vulnerability for the future (Table
1, \citet{Coyne2023} \& \citet{Skirrow2013} ):

\begingroup
\fontsize{7.5pt}{9.0pt}\selectfont
\setlength{\LTpost}{0mm}
\begin{longtable}{>{\centering\arraybackslash}p{\dimexpr 75.00pt -2\tabcolsep-1.5\arrayrulewidth}>{\centering\arraybackslash}p{\dimexpr 75.00pt -2\tabcolsep-1.5\arrayrulewidth}>{\centering\arraybackslash}p{\dimexpr 75.00pt -2\tabcolsep-1.5\arrayrulewidth}>{\centering\arraybackslash}p{\dimexpr 75.00pt -2\tabcolsep-1.5\arrayrulewidth}>{\centering\arraybackslash}p{\dimexpr 75.00pt -2\tabcolsep-1.5\arrayrulewidth}}
\caption*{
{\large Table 1: summary of crtical element in Australia}
} \\ 
\toprule
{CRITICAL ELEMENT} & {PRODUCTION(KT)} & \% of Global production & {ORE RESERVE(KT)} & {USAGE} \\ 
\midrule\addlinespace[2.5pt]
Aluminum \& derivative(Al) & 20 & 14\% & 1,160 & Aerospace sector and coating in Li-ion batteries \\ 
Cobalt(Co) & 5.9 & 3\% & 613 & Li-ion battery cathodes,superalloys, steel, and magnets \\ 
Gallium(Ga) & - & - & - & ICs, GaN laser diodes and photovoltaics films \\ 
Germanium(Ge) & - & - & - & Fiber/infared optics,Polymerization Catalysts and semiconductors \\ 
Lithium(Li) & 61 & 47\% & 4,563 & Li-ion batteries and Lithium carbonate and lithium oxide in glass or ceramics \\ 
Magnesium(Mg) & 2.6 & 10\% &  Not available & Pyrotechnics, medicine and nanocomposites in automotive/aerospace \\ 
Manganese(Mn) & 3.3 & 17\% & 120,000 & Steel,Agricultral fertiliser and dry cell batteries \\ 
Nickel(Ni) & 150 & 4.5\% & 20.4\textsuperscript{\textit{1}} & Non-ferrous alloys,magnets and electroplating \\ 
Rare-earth elements(REE)\textsuperscript{\textit{2}} & 18 & 6\% & 3,121 & Catalyst,magnets and polishing \\ 
Silicon(Si) & 0.05 & 1\% & - & Silicon wafer in electronic \& photovoltaic cells and synthetic polymers \\ 
Tantalum(Ta) & 0.057 & 3\% & 49.8 & Micro-capacitors and medical technology \\ 
Titanium(Ti) & 0.85 & 8.4\% & 82,100 & Aerospace and marine application or pigment through Titanium dioxide \\ 
Tungsten(W) & - & - & 213 & Lightning and chemical compunds \\ 
Vanadium(V) & - & - & 2,948 & Steel alloys \\ 
Zirconium(Zr) & 0.5 & 36\% & 29,200 & Cladding fuel rods in nuclear reactors \\ 
\bottomrule
\end{longtable}
\begin{minipage}{\linewidth}
\textsuperscript{\textit{1}}data collected in 2012.\\
\textsuperscript{\textit{2}}17 elements, including lanthanoid,Scandium(Sc) and Yttrium(Y).\\
Source: Skirrow et al., 2013 \& Coyne and Campbell, 2023\\
\end{minipage}
\endgroup

Rare-earth elements,a significant branch in critical element, comprise
of 15 element in lanthanoid family and 2 extra elements with similar
chemical properties--Scandium(Sc) and Yttrium(Y). Unlike name
implication, albeit their overall natural abundance(NA) in earth crust
is not extremely rare (average 180-200 ppm) their distribution in earth
is quite scattered and strong propensity to coexist in pairs or group
within ore deposits in terms of geochemical properties. Therefore, only
few REE can be concentrated to a degree that permits for commercial
mining. The largest segment of global consumption is catalyst(24\%) in
automotive or petrochemical industry, as well as equally high
demand(23\%) in magnet(NdFeB, SmCo etc.) \citet{Zhou2017}.

With the exuberant demand shock from EVs and clean energy exploration,
the consumption of critical minerals is foreseeable to be thrived:
Geoscience Australia estimates that by 2040, critical elements related
to lithium batteries (such as Li, Co, and Ni) will see an exponential
surge of 20 to 40 times, while REEs (rare earth elements) will see a
7-fold increase during the same period.

\hypertarget{critical-elements-in-coal}{%
\section{Critical elements in coal}\label{critical-elements-in-coal}}

Answer the following question:

\begin{enumerate}
\def\labelenumi{\arabic{enumi}.}
\setcounter{enumi}{1}
\tightlist
\item
  What types of critical elements are found in coal, and in what
  concentrations? In an ideal state, any chemical elements can be
  detected in coal deposit, which may originate from syn-genetic plant
  decays or epigenetic source after . Its concentration will be
  determined on coalified compaction and the heat \& pressure
  environment.
\end{enumerate}

In \citet{Hodgkinson2020} element mapping project on Bowen basin,the
largest coal reserves in Australia, the concentration of element
composition is subjective to sample's lithology rather than the depth
grading:

1). In coal and derivative, albeit majority of element concentrations is
inferior of the benchmark against earth crust average, local samples
exhibit enrichment in HREE and Scandium in respect to
\citet{McLennan2011} Post-Archaean Australian Shales (PAAS) standard ,
while abnormal 4-6 times higher than crustal average in moderate
critical element, Bismuth(Bi) .

2). Siltstone and mudstone has a lackluster finding to classify
enrichment for majority of elements concentration,except for the
concentration of Cobalt compound barely meet crustal average, whose
ubiquitous economic value may warrant further examination.

3). As the sediment from volcanic ash, tuffaceous rock is rich in pumice
and lithic fragments. The sample display a series of elevated
concentrations of strategic elements including REE, Ga and Bi. Besides,
a potential Lithium-rich borehole is found, with approximate 5 times
higher than crustal average.

Below is the summary result (Table 2) from the distribution of critical
element :

\begingroup
\fontsize{7.5pt}{9.0pt}\selectfont
\setlength{\LTpost}{0mm}
\begin{longtable}{>{\centering\arraybackslash}p{\dimexpr 75.00pt -2\tabcolsep-1.5\arrayrulewidth}>{\centering\arraybackslash}p{\dimexpr 75.00pt -2\tabcolsep-1.5\arrayrulewidth}>{\centering\arraybackslash}p{\dimexpr 75.00pt -2\tabcolsep-1.5\arrayrulewidth}>{\centering\arraybackslash}p{\dimexpr 75.00pt -2\tabcolsep-1.5\arrayrulewidth}>{\centering\arraybackslash}p{\dimexpr 75.00pt -2\tabcolsep-1.5\arrayrulewidth}}
\caption*{
{\large Table 2: summary of crtical element in Australia}
} \\ 
\toprule
{CRITICAL ELEMENT} & Crust\textsuperscript{\textit{1}} & {AVERAGE CONCENTRATION(PPM)} & {HIGEST CONCENTRATION(PPM)} & \% of above crustal average \\ 
\midrule\addlinespace[2.5pt]
\multicolumn{5}{>{\raggedright\arraybackslash}m{450pt}}{{Coal\& associate}} \\[2.5pt] 
\midrule\addlinespace[2.5pt]
Lithium & 21.0 & 13.70 & 25.0 & 22.22\% \\ 
REE & 184.0 & 115.80 & 205.0 & 11.11\% \\ 
Cobalt & 17.0 & 16.90 & 30.0 & 44.44\% \\ 
Nickel & 47.0 & 11.20 & 40.0 & 0.00\% \\ 
Tantalum & 1.0 & 0.33 & 1.0 & 33.33\% \\ 
Vanadium & 97.0 & 85.60 & 140.0 & 22.22\% \\ 
Zirconium & 193.0 & 102.10 & 160.0 & 0.00\% \\ 
Gallium & 17.0 & 12.30 & 25.0 & 44.44\% \\ 
Bismuth & 0.2 & {\bfseries 0.59} & 1.0 & 66.67\% \\ 
\midrule\addlinespace[2.5pt]
\multicolumn{5}{>{\raggedright\arraybackslash}m{450pt}}{{Siltstone \& mudstone}} \\[2.5pt] 
\midrule\addlinespace[2.5pt]
Lithium & 21.0 & 17.20 & 28.0 & 33.33\% \\ 
REE & 184.0 & 138.80 & 189.0 & 16.67\% \\ 
Cobalt & 17.0 & {\bfseries 39.30} & 134.0 & 66.67\% \\ 
Nickel & 47.0 & 25.80 & 73.0 & 33.33\% \\ 
Tantalum & 1.0 & 0.20 & 1.0 & 16.67\% \\ 
Vanadium & 97.0 & {\bfseries 102.00} & 225.0 & 33.33\% \\ 
Zirconium & 193.0 & 116.80 & 243.0 & 16.67\% \\ 
Gallium & 17.0 & 15.00 & 22.0 & 33.33\% \\ 
Bismuth & 0.2 & 0.18 & 0.4 & 66.67\% \\ 
\midrule\addlinespace[2.5pt]
\multicolumn{5}{>{\raggedright\arraybackslash}m{450pt}}{{Tuffaceous rocks}} \\[2.5pt] 
\midrule\addlinespace[2.5pt]
Lithium & 21.0 & {\bfseries 21.50} & 105.0 & 12.50\% \\ 
REE & 184.0 & {\bfseries 244.00} & 441.0 & 87.50\% \\ 
Cobalt & 17.0 & 9.25 & 19.0 & 12.50\% \\ 
Nickel & 47.0 & 3.75 & 30.0 & 0.00\% \\ 
Tantalum & 1.0 & {\bfseries 1.10} & 2.0 & 100.00\% \\ 
Vanadium & 97.0 & 27.90 & 70.0 & 0.00\% \\ 
Zirconium & 193.0 & 153.75 & 282.0 & 25.00\% \\ 
Gallium & 17.0 & {\bfseries 32.25} & 37.0 & 100.00\% \\ 
Bismuth & 0.2 & {\bfseries 0.71} & 1.2 & 100.00\% \\ 
\bottomrule
\end{longtable}
\begin{minipage}{\linewidth}
\textsuperscript{\textit{1}} McLennan 2011  Post-Archean
Australian Shale standard \\
Source: Hodginkson et al., 2020\\
\end{minipage}
\endgroup

\hypertarget{economic-concentrations}{%
\section{Economic concentrations}\label{economic-concentrations}}

Answer the following question:

\begin{enumerate}
\def\labelenumi{\arabic{enumi}.}
\setcounter{enumi}{2}
\tightlist
\item
  What concentrations of critical elements (including rare earth
  elements) would be considered economic to extract from coal?
\end{enumerate}

The booming demand for critical element, has not merely driven interest
in alternative sources beyond traditional mining but also stimulate the
exploration of operational feasibility from an economic value
perspective. Economic assessment is hinged with the priority concern on
concentrations, as well as demand-supply dynamic and operational cost
\citet{usde2017}.

In general, \citet{Seredin2012} study for the coal mining in Far east
region, Russia manifest the cut-off criterion for REE extraction will
range from 800-1000 ppm, subject to seam thickness of the coal. However,
the primitive cut-off grading does not correct reflect market trend on
each individual composition. The new outlook coefficient(Table 3) is
introduced to lever the economic degree of rarity for REY oxide(REO).
Elements with plentiful deposits or relatively narrow application
markets will be graded as `Excessive', vice versa to be graded as
`Vital'. The ratio formula can be written as:

\[ \tag{1} C_{Outlook}=\frac{\sum{C_{Vital}}}{\sum{C_{Excessive}}}  \]
\[ \tag{2} C_{\text{Critical percent}}=\frac{\sum{C_{Vital}}}{\sum{C_{REE}}} \]

\begingroup
\fontsize{7.5pt}{9.0pt}\selectfont
\setlength{\LTpost}{0mm}
\begin{longtable}{>{\centering\arraybackslash}p{\dimexpr 75.00pt -2\tabcolsep-1.5\arrayrulewidth}>{\centering\arraybackslash}p{\dimexpr 75.00pt -2\tabcolsep-1.5\arrayrulewidth}>{\centering\arraybackslash}p{\dimexpr 75.00pt -2\tabcolsep-1.5\arrayrulewidth}}
\caption*{
{\large Table 3: summary of crtical element in Australia}
} \\ 
\toprule
{REE} & Category\textsuperscript{\textit{1}} & {SIGNIFICANCE} \\ 
\midrule\addlinespace[2.5pt]
Scandium(Sc) & - & -\textsuperscript{\textit{2}} \\ 
Yttrium(Y) & M & Vital \\ 
Lanthanum (La) & L & Moderate \\ 
Cerium (Ce) & L & Excessive \\ 
Praseodymium (Pr) & L & Moderate \\ 
Neodymium (Nd) & L & Vital \\ 
Promethium (Pm) & - & -\textsuperscript{\textit{2}} \\ 
Samarium (Sm) & L & Moderate \\ 
Europium (Eu) & M & Vital \\ 
Gadolinium (Gd) & M & Moderate \\ 
Terbium (Tb) & M & Vital \\ 
Dysprosium (Dy) & M & Vital \\ 
Holmium (Ho) & H & Excessive \\ 
Erbium (Er) & H & Vital \\ 
Thulium (Tm) & H & Excessive \\ 
Ytterbium (Yb) & H & Excessive \\ 
Lutetium (Lu) & H & Excessive \\ 
\bottomrule
\end{longtable}
\begin{minipage}{\linewidth}
\textsuperscript{\textit{1}}L- Light; M- Middle; H- Heavy\\
\textsuperscript{\textit{2}}Not included due to heterogeneous and radioactive reason.\\
Source: Seredin et al., 2012\\
\end{minipage}
\endgroup

\citet{Ketris2009} research measured average trace REE concentration
globally at 403.5 ppm, with an approximately 1 outlook coefficient and
35\% \(C_{\text{Critical percent}}\). Continual research by
\citet{Choudhary2022} analyzed approximately 288 coal,fly ash samples,
reporting 11 of 13 nations has been detected above-average case,
especially notably enrichment in China, Russia and Central Asia.

Furthermore, \citet{DaiFinkelman2018} also compiled the cut-off
grade(Table 4) for key elements in coal.

\begingroup
\fontsize{7.5pt}{9.0pt}\selectfont
\setlength{\LTpost}{0mm}
\begin{longtable}{>{\centering\arraybackslash}p{\dimexpr 75.00pt -2\tabcolsep-1.5\arrayrulewidth}>{\centering\arraybackslash}p{\dimexpr 75.00pt -2\tabcolsep-1.5\arrayrulewidth}}
\caption*{
{\large Table 4: summary of crtical element in Australia}
} \\ 
\toprule
{ELEMENT} & {CUT-OFF GRADE} \\ 
\midrule\addlinespace[2.5pt]
Uranium(U) & 1000 \\ 
Germanium(Ge)* & 300 \\ 
Vanadium(V)* & 1000 \\ 
Scandium(Sc)* & 100 \\ 
Selenium(Se) & 500-800 \\ 
Niobium(Nb) & 300 \\ 
Zirconium(Zr)* & 2000 \\ 
Molybdenum(Mo) & 1000 \\ 
Rhenium(Re) & 1 \\ 
Tungsten(W)*\textsuperscript{\textit{1}} & 1000 \\ 
Antimony(Sb) & 100 \\ 
Beryllium(Be) & 300 \\ 
\bottomrule
\end{longtable}
\begin{minipage}{\linewidth}
\textsuperscript{\textit{1}}High-demand critical element\\
Source: Dai et al., 2018\\
\end{minipage}
\endgroup

\hypertarget{existing-economic-deposits}{%
\section{Existing economic deposits}\label{existing-economic-deposits}}

Answer the following question:

\begin{enumerate}
\def\labelenumi{\arabic{enumi}.}
\setcounter{enumi}{3}
\tightlist
\item
  Are there any existing coal mines, or coal basins that are extracting
  critical elements from coal, or coal tailings?
\end{enumerate}

The extraction of critical elements from coal mines is still in the
observation and exploration stage. Industrial-grade REE and lithium
deposits have been respectively probed in the Jungar basin in Inner
Mongolia and the Ningwu basin in Shanxi Province, China. \citet{sun2010}
study on the Antaibao surface coalfield(Ningwu basin) found Lithium
enrichment with 10 times higher than China WAV in roof rock, floor rock
and coal seams. Another independent anlysis \citet{Dai2012} of
mineralogical and geochemical compositions in Daqingshan
coalfield(Jungar basin) exhibits L-type or H-type of REY ,Ga and
\(Al_{2}O_3\) abnormalities in coal bench, widely distributed in
chlorite,kaolinite and goyazite.

\hypertarget{extraction-of-critical-elements}{%
\section{Extraction of critical
elements}\label{extraction-of-critical-elements}}

Answer the following question:

\begin{enumerate}
\def\labelenumi{\arabic{enumi}.}
\setcounter{enumi}{4}
\tightlist
\item
  What analytical and processing methods can be used to extract critical
  elements from coal?
\end{enumerate}

The analytic methods in critical element mapping and quantification can
be summarized as \citet{Eterigho2021}:

1). \textbf{Proximate analysis}: As the standard practice of ASTM D3172,
it incorporates the composition breakdown through the learning of
moisture,volatile matter, ash and carbon content attribute.

2). \textbf{X-Ray Fluorescence (XRF) analysis}: As the standard practice
of ASTM D4326-13, it incorporate spectrometric identification and
quantification of the concentration of the critical element.

3). \textbf{Inductively Coupled Plasma Atomic Emission Spectrometry
(ICP-AES)}:As the standard practice of ASTM D 6357--19,it is
particularly utilized to determine the occurrence and concentration for
trace elements e.g.~arsenic, cadmium, mercury, lead, and other
potentially hazardous or valuable elements.

4). \textbf{Laser-induced breakdown spectroscopy (LIBS)}:A short pulse
laser is focused on the sample surface to form a plasma, and then the
plasma emission spectrum is analyzed to determine the material
composition and content of the sample.

5). \textbf{Transmission Electron Microscopy(TEM)}: It is a is an
advanced diagnostic technique used to observe material structures at the
nanometer scale. With transmitting a beam of electrons through an
ultra-thin specimen, it can return an image that reveals its internal
structure and pattern.

6). \textbf{Intelligent Scanning Electron Microscopy (SEM)}: Similar to
TEM, SEM is a computer-controlled scanning for coal and ores's surface
morphology and topography that reveal the texture, grain structure, and
other attribute in the sample.

7). \textbf{Chemical fractionation}: It refers to the process of
seperating mixture of mineral matters into different phrase, which can
help to target isolating elements or compounds.

The traditional techniques of element extraction primarily focus on

1). \textbf{acid-alkine} reagent to dissolve impurities in compound
mineral matters. Although this approach can achieve 70\%-80\% recover
rate, recently it has been controversial due to the excessive waste of
acid and alkaline solvents and the production of hazardous pollutants.

2). \textbf{Biological leaching} is another relatively
environment-beneficial approach that allow extraction under mild
reacting conditions and high operating safety.However, it encountered
the bottleneck of strain cultivation and microbial control, incurring
the scale-up challenge to industrial-level production.

Recently, \textbf{water-leaching} is a prevailing approach since it
mitigate the flaw from traditional techniques. Water leaching can be
considered as less environmentally detrimental compared to strong
acid/alkaline leaching, as well as cost effective for solvent selection.
The crucial stages on this preparation workflow are \textbf{low
temperature activation} and \textbf{water leaching}. During the stage of
low temperature activation, the chemical reaction within coal fly
ash(CFA) will be facilitated by complexation agents(ammonium salts or
weak acids) in covered alumina crucibles, which help liberate critical
elements from the matrix of the CFA. After the activation and cool down
to ambient temperature, the tablets are placed in water for the leaching
and dissolve process. Water acts as the leaching solvent, extracting
these soluble elements into the leachate.The configuration in
temperature and mass ratio of solvent will be the vital determinant for
optimized recovery. Take Lithium example, it can achieve a stable
leaching efficiency of 90\% through ammonium fluoride leaching at 150°C
with a \(SiO_2/NH_4F\) mass ratio of 1:1.35 \citet{Xu2021}.

Another innovation is Hydrophobic-Hydrophilic Separation (HHS), designed
to leverages the disparity of affinity( water-repellent
\&water-friendly) properties of substances to achieve separation.It can
treat as a complemetary application for small particle delamination
without size limit, providing flexible and extensible purpose in the
segregation of ultrafine coal \citet{Hodgkinson2021}.

\renewcommand\refname{References}
\bibliography{mybibfile.bib}


\end{document}
